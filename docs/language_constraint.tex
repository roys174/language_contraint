%
% File acl2017.tex
%
%% Based on the style files for ACL-2015, with some improvements
%%  taken from the NAACL-2016 style
%% Based on the style files for ACL-2014, which were, in turn,
%% based on ACL-2013, ACL-2012, ACL-2011, ACL-2010, ACL-IJCNLP-2009,
%% EACL-2009, IJCNLP-2008...
%% Based on the style files for EACL 2006 by 
%%e.agirre@ehu.es or Sergi.Balari@uab.es
%% and that of ACL 08 by Joakim Nivre and Noah Smith

\documentclass[11pt,a4paper]{article}
\usepackage[hyperref]{acl2017}
\usepackage{times}
\usepackage{latexsym}

\usepackage{url}

%\aclfinalcopy % Uncomment this line for the final submission
%\def\aclpaperid{***} %  Enter the acl Paper ID here

%\setlength\titlebox{5cm}
% You can expand the titlebox if you need extra space
% to show all the authors. Please do not make the titlebox
% smaller than 5cm (the original size); we will check this
% in the camera-ready version and ask you to change it back.

\newcommand\BibTeX{B{\sc ib}\TeX}

\title{???}

\author{First Author \\
  Affiliation / Address line 1 \\
  Affiliation / Address line 2 \\
  Affiliation / Address line 3 \\
  {\tt email@domain} \\\And
  Second Author \\
  Affiliation / Address line 1 \\
  Affiliation / Address line 2 \\
  Affiliation / Address line 3 \\
  {\tt email@domain} \\}

\date{}

\begin{document}
\maketitle
\begin{abstract}
People's writing style is affected by many factors, including topics, sentiment, and individual personality. 
In this paper we show that writing tasks that impose constraints on the writer result in the author adopting a  different writing style compared to tasks that do not.
As a case study, we experiment with a recently published machine reading task: the story cloze task \cite{Mostafazadeh:2016}. 
In this task, annotators were asked to generate two sentences: one which makes sense given a previous paragraph and another which doesn't.
We show that a linear classifier, which applies only simple style features, such as sentence length and PoS counts, obtains state-of-the-art results on the task,
substantially higher than sophisticated sequence-to-sequence models.
Importantly, our model doesn't even look at the previous paragraph, just the two candidate sentences, which, out of context, differ only in the constraint put on the authors. 
Our results indicate that such constraints dramatically affect the way people write. 
They also suggest that careful attention to the instructions given to the authors needs to taken when designing new NLP tasks.

  \end{abstract}

% include your own bib file like this:
%\bibliographystyle{acl}
%\bibliography{acl2017}
\bibliography{acl2017}
\bibliographystyle{acl_natbib}

\end{document}
